% Preamble
%
\documentclass[a4paper,09pt]{article}

\usepackage[familydefault,light]{Chivo}
\usepackage{stmaryrd}
\usepackage[utf8]{inputenc}
\usepackage{xcolor}
\usepackage[pdftex]{graphicx}
\usepackage[english]{babel}
\usepackage{float}
\usepackage{hyperref}
\hypersetup{colorlinks=true,linkcolor=magenta}

% Commands
%
%% So we can highlight things (YBG is for Yellow BackGround, if you want to
%% create new ones - power to the colours ! - follow the convention ;)

\newcommand{\YBG}[1]{\colorbox[rgb]{1,1,0.3}{#1}}

% Document
%

\setlength{\parindent}{0cm}
\author{Elie Canonici Merle, Colin González, Sylvain Ribstein, \\
  Kevin Sanchis, Anas Aarab, Paul Laforgue}

\title{Software Requirements Specification}

\date{October 17, 2016 \\ \small Last updated : \today}

\begin{document}

\maketitle

\begin{abstract}
  This file provides an overview of the requirements for the project of the
  course POCA 2016.
\end{abstract}

\section*{Introduction}

\paragraph{Purpose}

The goal of the project is to develop a framework in SCALA to implement games
such as pokemon-go (i.e a MMORPG plus augmented reality).
The project is open source.

\paragraph{Definitions}
Augmented reality : the concept here is still vague, the main importance is
to make use, eventually, of a real life map, i.e. \textit{à la} Google Maps, in
which a player is able to move through the World.\\
MMORPG : Massively Multiplayer Online Role-Playing Game.
We expect the software to manage several (to be determined later) users at a time.

\paragraph{System overview}
In this document we detail the specification of the underlying engine and the
client of the software. The engine represents an abstraction of the game, the
gameplay and its features. On top of the engine a more precise implementation
is to be added. The implementation provides the setup, as well as a gameplay
and the features expected by the player. The engine itself is a server
application able to manage the clients (players) connected on it. The server is
expected to resolve concurrency issues to ensure safety and liveness in order to
prevent lost updates as well as insuring coherence between players.
For instance, the server is responsible of the information given to the player.
More precisely, if a token (i.e. a Pokémon) is spawned in a location $l$,
then all the clients (players) near enough to the position $l$ must be able to
interact with the tokens. On the other side the client is the part of the
sofware that allows a user to interact with the server application. The client
provides a user interface and simple documentation used as an accessible goto
reference for any kind of user. The developpement of such software, both client
and software, is supposed to be scalable. Therefore an object oriented and
modular paradigm is needed.

\newpage

\tableofcontents

\newpage

\section{Overall Description}

\subsection{Product Perspective}
The software presented in this document is a MMORPG game. The game will use
virtual reality features to provide a realistic experience of gameplay for its
users. This videogame is mainly inspired by the Pokémon Go mobile game.
In Pokémon Go a player goes on the hunt after Pokémons everywhere, possibly
all over the World. The player uses Pokeballs in order to capture such creatures.
The goals are to capture all existing kind of Pokémons and conquer and hold
particular spots in the real world. Pokémons are trainable creatures, they are
put to fight by their trainers in such way that the player earns experience from
won battles. Players are part of a team so that players from the same team
can help each other during fights. The software is expected to have
similar features.

\subsection{Design Constraints}
The software must absolutely respect the specifications on figure \ref{descons}.
\begin{figure}[!h]
  \begin{center}
    \begin{enumerate}
    \item The game is about augmented reality Pokémon-like characters
      with statistics about their level, health points and other statistics.
    \item A map like model allowing geolicalisation. In the first version
      a simple model is acceptable. However it is desirable to add Google
      Maps like World Map.
    \item The game makes use of augmented reality or real life objects
      (i.e. Pokéballs, Pokéstops, points of interest, Pokémon centers and
      the like).
    \item Interaction between several players.
    \item A modular architecture in order to provide extra features
      using extensions.
    \item The software must be implemented using the Scala language.
    \item The software must respect a distributed architecture using a
      server-client architecture.
    \item A protocol for the server-client must be defined and used mandatorily.
      No off protocol communications should be accepted.
    \item The synchronization of the client is achieved through the server part.
    \item The engine must not need high resources (CPU and memory) so that it
      is playable in a variety of devices.
    \item The software is easy to use by any sort of player.
    \item Optmizations are desireable. As an example, characters should not
      be spawned in useless places. Particularly, a place in the world with
      no players near enough.
    \item The client software doens't store the user data, the client
      retrieves it when the user signs in.
    \end{enumerate}
    \caption{Design Constraints}
    \label{descons}
  \end{center}
\end{figure}
\subsection{Product Functions}
The software must ensure the functions on figure \ref{prodfuns}.
\begin{figure}[H]
  \begin{center}
    \begin{enumerate}
    \item The software provides a means to interact with virtual characters.
    \item The software functions in real-time.
    \item The software is secure.
    \item New players can register to game.
    \item The software  manages the conection of clients to the server.
    \item The software ensures a dynamic gameplay.
    \item The game difficulty is low and mostly based and random events
      and progression by regular sessions of gameplay.
    \item The software may offer interaction with other between players.
    \end{enumerate}
    \caption{Product Functions}
    \label{prodfuns}
  \end{center}
\end{figure}

\subsection{User characteristics}

The user has to be connected to the server (and by extension to internet) in
order to play the game. There is no offline version. Since this MMORPG is based
on augmented reality, the user needs to use a device allowing him to be located
and to move freely, for example a smartphone.

\section{Specific requirements}

\subsection{Performance requirements}

The application has to be constantly connected with the user and remember its
current information.
In that sense, if the client-side crashes, the application should be able
to restore its previous state.
The player should have the possibility to look for its progression and
so access databases at anytime.

\subsection{Logical database requirement}

The database should distinguish between static (age of the user,..)
and dynamic informations (its current location,..).\\
Briefly, the database will need to store numerous informations about the:

\begin{itemize}
\item
  \textbf{Users}: name, age, contacts, friends,
  inventory, current and last locations...
\item
  \textbf{Pokemons}: name, race, characteristics, color...
\item
  \textbf{Items}: name, price...
\end{itemize}

An adequate object-relational database choice would be: PostgreSQL.
It's open source and has a good reputation.
The relational model has to be scalable, and should be able to
support a lot of players. Also, it should be easily extended.

\end{document}
