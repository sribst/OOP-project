\usetikzlibrary{arrows}

% to set size for itemize in node
\usepackage{varwidth}

% to reduce left margin of itemize
\usepackage{enumitem}
\setlist{leftmargin=0mm}

\tikzset{
  class/.style={
    rectangle,
    rounded corners,
    draw=black, very thick,
    minimum height=2em,
    inner sep=4pt,
    text centered,
  },
}

% use for the 'parser'
\usepackage{etoolbox}

% second try of foreach loop
\newcommand\handler[1]{\item[-] #1}
\DeclareListParser*\forsemicolonlist{;}

% provides \isempty test
\usepackage{xifthen}

\usepackage{enumitem}
% \newclass{class name}{attribute <list>}{method <list>}
% attribute list and method list are list with ',' as separator
\newcommand{\newclass}[3]{
  {
    \begin{tabular}{l}
     \renewcommand{\arraystretch}{1.5}
      \large{\textbf{#1}}
      \ifthenelse{\isempty{#2}}{}{
      \\
      \hline
      \textbf{Attribute}\\
      {\begin{varwidth}{\linewidth}\begin{itemize}[noitemsep]
            \forsemicolonlist\handler{#2}
          \end{itemize}\end{varwidth}}
      }
      \ifthenelse{\isempty{#3}}{}{
      \\
      \hline
      \textbf{Method}\\
      {\begin{varwidth}{\linewidth}\begin{itemize}[noitemsep]
            \forsemicolonlist\handler{#3}
          \end{itemize}\end{varwidth}}
      }
   \end{tabular}}
}

\newcommand{\newtable}[2]{
  {
    \begin{tabular}{l}
     \renewcommand{\arraystretch}{1.5}
      \large{\textbf{#1}}
      \ifthenelse{\isempty{#2}}{}{
      \\
      \hline
      {\begin{varwidth}{\linewidth}\begin{itemize}[noitemsep]
            \forsemicolonlist\handler{#2}
          \end{itemize}\end{varwidth}}
      }
   \end{tabular}}
}
